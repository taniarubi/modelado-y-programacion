\documentclass[letterpaper,11pt]{article}

% Soporte para los acentos.
\usepackage[utf8]{inputenc}
\usepackage[T1]{fontenc}    
% Idioma español.
\usepackage[spanish,mexico, es-tabla]{babel}
% Soporte de símbolos adicionales (matemáticas)
\usepackage{multirow}
\usepackage{amsmath}		
\usepackage{amssymb}		
\usepackage{amsthm}
\usepackage{amsfonts}
\usepackage{latexsym}
\usepackage{enumerate}
\usepackage{ragged2e}
\usepackage{hyperref}
% Modificamos los márgenes del documento.
\usepackage[lmargin=2cm,rmargin=2cm,top=2cm,bottom=2cm]{geometry}

\title{Facultad de Ciencias, UNAM \\ Modelado y Programación \\ Práctica 2}
\author{Hernández Morales José Ángel \\ No. de cuenta: 315137903 \\ 
        Rubí Rojas Tania Michelle \\ No. de cuenta: 315121719}
\date{07 de agosto de 2020}

\begin{document}
\maketitle

\begin{enumerate}
    % Parte teórica.
    \item Menciona los principios de diseño esenciales de los patrones Decorator 
    y Adapter. Menciona una desventaja de cada patrón.

    \textsc{Solución:}
    \begin{enumerate}
        \item Patrón Decorator 
        
        Permite agregar nuevas funcionalidades a las clases sin modificar su 
        estructura. El concepto de este patrón es agregar de forma dinámica 
        nuevo comportamiento o funcionalidades a la clase principal, es decir, 
        vamos a \textit{decorar} los objetos para darles más funcionalidad de 
        la que tienen en un principio.

        Una desventaja es la mantenibilidad del código ya que este patrón crea 
        muchos decoradores similares que a veces son difíciles de mantener y 
        distinguir. 

        \item Patrón Adapter
        
        Este patrón sirve cuando tenemos interfaces diferentes o incompatibles 
        entre sí y necesitamos que el cliente pueda usar ambas del mismo 
        modo. Adapter convierte las interfaces existentes en una nueva interfaz 
        para lograr la compatibilidad y la reutilización de las clases no 
        relacionadas en una aplicación.

        Una desventaja es que a veces se requieren muchas adaptaciones a lo 
        largo de una cadena de adaptadores para alcanzar el tipo que se 
        requiere. Otra podría ser que este patrón aumenta innecesariamente 
        el tamaño del código, ya que la herencia de clases se utiliza menos 
        y mucho código se duplica innecesariamente entre las clases. 
    \end{enumerate}

    % Funcionalidad del programa.
    \item Funcionalidad del programa
    \begin{enumerate}
        \item Primero, debemos posicionarnos dentro de la carpeta \textbf{src}.
        Aquí se encuentran todas las clases de nuestra práctica.

        \item Compilamos todas nuestras clases.
        \begin{verbatim}
            $ javac *.java
        \end{verbatim}

        \item Ejecutamos nuestro programa principal.
        \begin{verbatim}
            $ java Main 
        \end{verbatim}
    \end{enumerate}

    \begin{thebibliography} {9}
        \bibitem{1}
        \url{https://experto.dev/patron-de-diseno-decorator-en-java/}

        \bibitem{2}
        \url{https://www.codeproject.com/tips/468951/decorator-design-pattern-in-java}
        
        \bibitem{3}
        \url{https://www.genbeta.com/desarrollo/patrones-de-diseno-decorator}

        \bibitem{4}
        \url{https://reactiveprogramming.io/blog/es/patrones-de-diseno/adapter}

        \bibitem{5}
        \url{https://experto.dev/patron-de-diseno-adapter-en-java/}

        \bibitem{6}
        \url{https://javarevealed.wordpress.com/tag/adapter-pattern/}

        \bibitem{7}
        \url{https://javarevealed.wordpress.com/tag/adapter-pattern/}
    \end{thebibliography}
\end{enumerate}
\end{document}