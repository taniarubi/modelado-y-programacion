\documentclass[letterpaper,11pt]{article}

% Soporte para los acentos.
\usepackage[utf8]{inputenc}
\usepackage[T1]{fontenc}    
% Idioma español.
\usepackage[spanish,mexico, es-tabla]{babel}
% Soporte de símbolos adicionales (matemáticas)
\usepackage{multirow}
\usepackage{amsmath}		
\usepackage{amssymb}		
\usepackage{amsthm}
\usepackage{amsfonts}
\usepackage{latexsym}
\usepackage{enumerate}
\usepackage{ragged2e}
\usepackage{hyperref}
% Modificamos los márgenes del documento.
\usepackage[lmargin=2cm,rmargin=2cm,top=2cm,bottom=2cm]{geometry}

\title{Facultad de Ciencias, UNAM \\ Modelado y Programación \\ Práctica 2}
\author{Hernández Morales José Ángel \\ No. de cuenta: 315137903 \\ 
        Rubí Rojas Tania Michelle \\ No. de cuenta: 315121719}
\date{07 de agosto de 2020}

\begin{document}
\maketitle

\begin{enumerate}
    % Parte teórica.
    \item Menciona los principios de diseño esenciales de los patrones Decorator 
    y Adapter. Menciona una desventaja de cada patrón.

    \textsc{Solución:}
    \begin{enumerate}
        \item Patrón Decorator 
        
        Permite agregar nuevas funcionalidades a las clases sin modificar su 
        estructura. El concepto de este patrón es agregar de forma dinámica 
        nuevo comportamiento o funcionalidades a la clase principal. 

        Una desventaja es la mantenibilidad del código ya que este patrón crea 
        muchos decoradores similares que a veces son difíciles de mantener y 
        distinguir. 

        \item Patrón Adapter
    \end{enumerate}

    % Funcionalidad del programa.
    \item Funcionalidad del programa
    \begin{enumerate}
        \item Primero, debemos posicionarnos dentro de la carpeta \textbf{src}.
        Aquí se encuentran todas las clases de nuestra práctica.

        \item Compilamos todas nuestras clases.
        \begin{verbatim}
            $ javac *.java
        \end{verbatim}

        \item Ejecutamos nuestro programa principal.
        \begin{verbatim}
            $ java Main 
        \end{verbatim}
    \end{enumerate}

    \begin{thebibliography} {9}
        \bibitem{1}
        \url{https://experto.dev/patron-de-diseno-decorator-en-java/}

        \bibitem{2}
        \url{https://www.codeproject.com/tips/468951/decorator-design-pattern-in-java}
        
    \end{thebibliography}
\end{enumerate}
\end{document}