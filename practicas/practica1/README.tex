\documentclass[letterpaper,11pt]{article}

% Soporte para los acentos.
\usepackage[utf8]{inputenc}
\usepackage[T1]{fontenc}    
% Idioma español.
\usepackage[spanish,mexico, es-tabla]{babel}
% Soporte de símbolos adicionales (matemáticas)
\usepackage{multirow}
\usepackage{amsmath}		
\usepackage{amssymb}		
\usepackage{amsthm}
\usepackage{amsfonts}
\usepackage{latexsym}
\usepackage{enumerate}
\usepackage{ragged2e}
\usepackage{hyperref}
% Modificamos los márgenes del documento.
\usepackage[lmargin=2cm,rmargin=2cm,top=2cm,bottom=2cm]{geometry}

\title{Facultad de Ciencias, UNAM \\ Modelado y Programación \\ Práctica 1}
\author{Hernández Morales José Ángel \\ No. de cuenta: 315137903 \\ 
        Rubí Rojas Tania Michelle \\ No. de cuenta: 315121719}
\date{31 de julio de 2020}

\begin{document}
\maketitle

\begin{enumerate}
    % Parte teórica. 
    \item Menciona los principios de diseño esenciales de los patrones Observer
    y Strategy. Menciona una desventaja de cada patrón. 

    \textsc{Solución:}
    \begin{itemize}
        \item Patrón Observer
        
        El patrón Observer puede ser utilizado cuando hay objetos que dependen 
        de otro, necesitando ser notificados en caso de que se produzca algún 
        cambio en él. Su objetivo es definir una dependencia uno a muchos entre
        objetos, de tal forma que cuando el objeto cambie de estado, todos sus 
        objetos dependientes sean notificados automáticamente.

        Una desventaja de este patrón es la fuga de memoria, esto se debe a que 
        para agregar o quitar observadores se tiene que hacer de manera explícita.
        Aunque no se utilicen más los observadores sigue quedando la referencia 
        para notificarlo y esto hace que el recolector de basura no lo elimine. 

        \item Patrón Strategy 

        El patrón Strategy ayuda a definir diferentes comportamientos o 
        funcionalidades que pueden ser cambiadas en tiempo de ejecución. En el 
        patrón Strategy creamos diferetes clases que representan estrategias y 
        que podremos usar según alguna variación o input.

        Una desventaja de este patrón es que si sólo tenemos un par de 
        estrategias a implementar, puede ser un poco exagerado o excesivo
        implementar este patrón, ya que complicamos en exceso la casuística. 
        Otra desventaja podría ser que el cliente debe conocer todas las
        diferencias de las estrategias. 
    \end{itemize}

    % Funcionalidad del prograna.
    \item Funcionalidad del programa
    
    \begin{enumerate}
        \item Primero, debemos posicionarnos dentro de la carpeta \textbf{src}.
        Aquí se encuentran todas las clases de nuestra práctica.

        \item Compilamos todas nuestras clases.
        \begin{verbatim}
            $ javac *.java
        \end{verbatim}

        \item Ejecutamos nuestro programa principal.
        \begin{verbatim}
            $ java Main 
        \end{verbatim}
    \end{enumerate}

\end{enumerate}

\begin{thebibliography} {9}
    \bibitem{1}
    \url{https://giovanni-sanchez.blogspot.com/2009/05/observer.html}

    \bibitem{2}
    \url{https://somospnt.com/blog/155-patron-de-comportamiento-observer}

    \bibitem{3}
    \url{https://experto.dev/patron-de-diseno-strategy-en-java/}

    \bibitem{4}
    \url{https://dev.to/mangelsnc/patron-estrategia-strategy-pattern-4b29}
\end{thebibliography}

\end{document}
