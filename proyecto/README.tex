\documentclass[letterpaper,11pt]{article}

% Soporte para los acentos.
\usepackage[utf8]{inputenc}
\usepackage[T1]{fontenc}    
% Idioma español.
\usepackage[spanish,mexico, es-tabla]{babel}
% Soporte de símbolos adicionales (matemáticas)
\usepackage{multirow}
\usepackage{amsmath}		
\usepackage{amssymb}		
\usepackage{amsthm}
\usepackage{amsfonts}
\usepackage{latexsym}
\usepackage{enumerate}
\usepackage{ragged2e}
\usepackage{hyperref}
% Modificamos los márgenes del documento.
\usepackage[lmargin=2cm,rmargin=2cm,top=2cm,bottom=2cm]{geometry}

\title{Facultad de Ciencias, UNAM \\ Modelado y Programación \\ Proyecto Final}
\author{Hernández Morales José Ángel \\ No. de cuenta: 315137903 \\ 
        Rubí Rojas Tania Michelle \\ No. de cuenta: 315121719}
\date{04 de septiembre de 2020}

\begin{document}
\maketitle

\begin{enumerate}
    % Descripción del proyecto.
    \item Descripción del proyecto

    El proyecto final consiste en la simulación de un juego de soldaditos. Al 
    inicio del juego, podemos elegir entre tres diferentes ejércitos para poder 
    combatir a un enemigo. Después de que se nos despliega una pantalla con las 
    características de cada uno de nuestros soldados, inicia el juego. Tenemos 
    tres acciones disponibles:
    \begin{itemize}
        \item Atacar. Al seleccionar esta opción, los comandantes mandarán la 
        órden de atacar al enemigo. Si la distancia de los soldados con respecto 
        al enemigo es igual a $0$, entonces pueden atacar.

        \item Mover. Los soldados se encuentran a una distancia inicial del 
        enemigo, así que al seleccionar esta opción lo que pasa es que los 
        comandantes mandarán la órden de que los soldados se muevan en dirección 
        al enemigo.

        \item Reportar. Al seleccionar esta opción, los comandantes mandarán la 
        órden de reportarse a los soldados. Esto simplemente nos mostrará algunas 
        características de los soldados, como lo son el nombre, ID, puntos de vida,
        distancia con respecto al enemigo y el tipo de soldado que es.
    \end{itemize}

    Esta rutina se repetirá hasta que los puntos de vida del enemigo sean igual 
    a $0$. En este momento, el juego termina. Para jugar otra partida, bastará 
    con volver a ejecutar el programación principal.

    % Funcionalidad del programa.
    \item Funcionalidad del programa
    \begin{enumerate}
        \item Primero, debemos posicionarnos dentro de la carpeta \textbf{src}.
        Aquí se encuentran todas las clases de nuestro proyecto.

        \item Compilamos todas nuestras clases.
        \begin{verbatim}
            $ javac *.java
        \end{verbatim}

        \item Ejecutamos nuestro programa principal.
        \begin{verbatim}
            $ java Main 
        \end{verbatim}
    \end{enumerate}

    % Patrones de diseño utilizados.
    \item Patrones de diseño utilizados
    \begin{itemize}
        \item Factory
        
        Este patrón lo utilizamos para construir a cada uno de nuestros soldados.
        \item Composite
        \item Strategy 
        \item Observer 
    \end{itemize}
\end{enumerate}

\end{document}
