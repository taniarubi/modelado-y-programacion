\documentclass[letterpaper,11pt]{article}

% Soporte para los acentos.
\usepackage[utf8]{inputenc}
\usepackage[T1]{fontenc}    
% Idioma español.
\usepackage[spanish,mexico, es-tabla]{babel}
% Soporte de símbolos adicionales (matemáticas)
\usepackage{multirow}
\usepackage{amsmath}		
\usepackage{amssymb}		
\usepackage{amsthm}
\usepackage{amsfonts}
\usepackage{latexsym}
\usepackage{enumerate}
\usepackage{ragged2e}
\usepackage{hyperref}
% Modificamos los márgenes del documento.
\usepackage[lmargin=2cm,rmargin=2cm,top=2cm,bottom=2cm]{geometry}

\title{Facultad de Ciencias, UNAM \\ Modelado y Programación \\ Proyecto Final}
\author{Hernández Morales José Ángel \\ No. de cuenta: 315137903 \\ 
        Rubí Rojas Tania Michelle \\ No. de cuenta: 315121719}
\date{04 de septiembre de 2020}

\begin{document}
\maketitle

\begin{enumerate}
    % Descripción del proyecto.
    \item Descripción del proyecto

    El proyecto final consiste en la simulación de un juego de soldaditos. Al 
    inicio del juego, podemos elegir entre tres diferentes ejércitos para poder 
    combatir a un enemigo. Después de que se nos despliega una pantalla con las 
    características de cada uno de nuestros soldados, inicia el juego. Tenemos 
    tres acciones disponibles:
    \begin{itemize}
        \item Atacar. Al seleccionar esta opción, los comandantes mandarán la 
        órden de atacar al enemigo. Si la distancia de los soldados con respecto 
        al enemigo es igual a $0$, entonces pueden atacar.

        \item Mover. Los soldados se encuentran a una distancia inicial del 
        enemigo, así que al seleccionar esta opción lo que pasa es que los 
        comandantes mandarán la órden de que los soldados se muevan en dirección 
        al enemigo.

        \item Reportar. Al seleccionar esta opción, los comandantes mandarán la 
        órden de reportarse a los soldados. Esto simplemente nos mostrará algunas 
        características de los soldados, como lo son el nombre, ID, puntos de vida,
        distancia con respecto al enemigo y el tipo de soldado que es.
    \end{itemize}

    Esta rutina se repetirá hasta que los puntos de vida del enemigo sean igual 
    a $0$. En este momento, el juego termina. Para jugar otra partida, bastará 
    con volver a ejecutar el programación principal.

    % Funcionalidad del programa.
    \item Funcionalidad del programa
    \begin{enumerate}
        \item Primero, debemos posicionarnos dentro de la carpeta \textbf{src}.
        Aquí se encuentran todas las clases de nuestro proyecto.

        \item Compilamos todas nuestras clases.
        \begin{verbatim}
            $ javac *.java
        \end{verbatim}

        \item Ejecutamos nuestro programa principal.
        \begin{verbatim}
            $ java Main 
        \end{verbatim}
    \end{enumerate}

    % Patrones de diseño utilizados.
    \item Patrones de diseño utilizados
    \begin{itemize}
        \item Factory
        
        Este patrón lo utilizamos para construir a cada uno de nuestros 
        soldados.

        % Patrón de diseño Composite.
        \item Composite

        Con ayuda de este patrón creamos y manejamos una jerarquía (en forma 
        de árbol) entre los comandantes y los soldados. Esto resuelve la 
        comunicación entre éstos dos. 

        La clase \texttt{Componente} en este caso es la interfaz 
        \texttt{Soldado}, la cual contiene el comportamiento mínimo que debería 
        tener un soldado. Ésta implementa cuatro clases: \texttt{Comandante}, 
        \texttt{DeArtillería}, \texttt{DeCaballería} y \texttt{DeInfantería}. 
        Fue diseñada así ya que todos ellos son soldados, y en particular, para 
        poder combinarla junto con el patrón \texttt{Strategy} en el caso de 
        los soldados subordinados. La clase \texttt{Composite} correspondería a 
        la clase \texttt{Comandante}, ya que éste es el de mayor jerarquía entre 
        los soldados. Puede agregar, eliminar y obtener la lista con todos los 
        soldados que tiene como subordinados. Gracias a cómo está implementada
        la interfaz \texttt{Soldado}, al realizar las acciones propias de un 
        soldado, éste hace que todos sus soldados subordinados realizen dicha 
        acción. Finalmente, la clase \texttt{Hoja} corresponde a las clases 
        \texttt{DeArtillería}, \texttt{DeCaballería} y \texttt{DeInfantería}
        ya que éstos no contienen a otros elementos.Así, cada uno realiza 
        la acción que les corresponde, y nada más.

        % Patrón de diseño Strategy.
        \item Strategy 

        Con ayuda de este patrón creamos un \texttt{Soldado} que pueda 
        comportarse de formas diferentes (lo cual se define en el momento 
        de su creación). Aunque todos los soldados realizarán las acciones 
        mínimas correspondientes a un \texttt{Soldado}, ellos lo hacen de 
        diferente modo, dependiendo de sí es un soldado \texttt{DeArtillería}, 
        un soldado \texttt{DeCaballería} o un soldado de \texttt{DeInfantería}.

        La interfaz \texttt{Estrategia} corresponde a la interfaz 
        \texttt{Soldado}. Ésta contiene el comportamiento mínimo que debe 
        realizar cada soldado. Las clases \texttt{EstrategiaConcretaN}
        corresponden a las clases \texttt{DeArtillería}, \texttt{DeCaballería}
        y \texttt{DeInfantería}. 

        % Patrón de diseño Observer.
        \item Observer

        Con ayuda de este patrón podemos notificar a los comandantes cuando 
        la acción que quiere realizar el usuario cambia. Esto resuelve la 
        comunicación entre el usuario y los comandantes (los cuales, a su vez,
        le comunicarán a sus soldados subordinados la acción que deseea el 
        usuario).

        Tenemos las interfaces \texttt{Sujeto} y \texttt{Observador}, las cuales 
        implementan las clases \texttt{Usuario} y \texttt{Comandante}, 
        respectivamente. La clase \texttt{Usuario} puede agregar, eliminar y 
        notificar a los observadores, que en este caso se tratan de los 
        comandantes. Éstos a su vez, logran notificar a los soldados la acción 
        que deben realizar, de acuerdo a los deseos del usuario.
    \end{itemize}

    Para la clase principal \texttt{Main} no utilizamos algún patrón de diseño 
    ya que considerámos que no era necesario.
\end{enumerate}
\end{document}
